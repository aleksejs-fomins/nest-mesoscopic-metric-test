\documentclass[10pt,a4paper,draft]{article}
\usepackage[utf8]{inputenc}
\usepackage{amsmath}
\usepackage{amsfonts}
\usepackage{amssymb}

\title{Analysis of accuracy of functional connectivity metrics for optogenetic data}
\author{Alekejs Fomins, Yaroslav Sych}

\begin{document}

\maketitle

\section{Functional Connectivity Metrics}
\subsection{Correlation}
\subsection{Spearmann Rank}
\subsection{Granger Causality}
\subsection{Transfer Entropy}

\section{Toy Examples}
\subsection{Simple Pendulums}
\subsection{Double Pendulum}
\subsection{Kuramoto Oscillators}

\section{Mesoscopic brain-like model}

\begin{enumerate}
  \item Construct a balanced exc-inh network
    \subitem Select fixed ratio of exc-inh neurons
    \subitem Sparse random or nearest-neighbor connectivity
    \subitem Random weights
  \item Construct a mesoscopic network from balanced networks
    \subitem Sparse random connections between some subregions
    \subitem Variable synaptic delay on long-range projections
  \item Model IF spiking activity
  \item Model sensory-like input
\end{enumerate}

\subsection{Approximate Data}

\begin{enumerate}
  \item Convert spikes to EPSPs
  \item Convolve EPSPs with $e^{-t/\tau_{GCaMP}}$
  \item Add random background fluorescence to each cell
  \item YARO: Approx ratio of neurons visible to fiber
  \item YARO: Compute fiber signal as random weighted average of neurons visible to fiber
  \item EXTRA: Effects of axons and dendrites in FOV
  \item EXTRA: If we choose to include \textbf{wide-field}, we must use geometric distribution of cells together with semi-accurate identification of sub-regions, which is much harder
\end{enumerate}

\subsection{Mean-Field Approximation?}

Attempt to approximate the network by a network of latent variables, and see if these variables can be estimated from the optical signal


\section{Analysis}




\end{document}